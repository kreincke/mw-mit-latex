% mycsrf cloak file
%
% (c) Karsten Reincke, Frankfurt a.M. 2010, 2011, ff.
%
% This file is licensed under the Creative Commons Attribution 3.0 Germany
% License (http://creativecommons.org/licenses/by/3.0/de/):
% For details see teh file LICENSE in the top directory
%
% select the document class
% S.26: [ 10pt|11pt|12pt onecolumn|twocolumn oneside|twoside notitlepage|titlepage final|draft
%         leqno fleqn openbib a4paper|a5paper|b5paper|letterpaper|legalpaper|executivepaper openrigth ]
% S.25: { article|report|book|letter ... }
%
% oder koma-skript S.10 + 16
\documentclass[
  DIV=calc,
  BCOR=5mm,
  12pt,
  headings=small,
  twoside,
  abstract=true,
  toc=bib,
  xcolor=dvipsnames,
  openany,
  english,ngerman]{scrbook}

%%% (1) general configurations %%%
\usepackage[utf8]{inputenc}

%%% (2) language specific configurations %%%
\usepackage[]{a4,babel}
\selectlanguage{ngerman}

% package for improving the grey value and the line feed handling
\usepackage{microtype}

%language specific quoting signs
\usepackage{csquotes}

% jurabib configuration
\usepackage[see]{jurabib}
\bibliographystyle{jurabib}
\input{cfg/inc.jbib-cfg.tex}

% language specific hyphenation
\input{cfg/inc.hyphenations.tex}

%%% (3) layout page configuration %%%

% select the visible parts of a page
% S.31: { plain|empty|headings|myheadings }
%\pagestyle{myheadings}
\pagestyle{headings}

% select the wished style of page-numbering
% S.32: { arabic,roman,Roman,alph,Alph }
\pagenumbering{arabic}
\setcounter{page}{1}

% select the wished distances using the general setlength order:
% S.34 { baselineskip| parskip | parindent }
% - general no indent for paragraphs
\setlength{\parindent}{0pt}
\setlength{\parskip}{1.2ex plus 0.2ex minus 0.2ex}


%%% (4) general package activation %%%
%\usepackage{utopia}
%\usepackage{courier}
%\usepackage{avant}
\usepackage[dvips]{epsfig}

% graphic

\usepackage{array}
\usepackage{shadow}
\usepackage{fancybox}

\usepackage{amsmath}
\usepackage{amsfonts}


\usepackage{chngcntr}


%- start(footnote-configuration)

\deffootnote[1.5em]{1.5em}{1.5em}{\textsuperscript{\thefootnotemark)\ }}

% if document class = book: count footnotes from start to end
\counterwithout{footnote}{chapter}
%- end(footnote-configuration)



%for using label as nameref
\usepackage{nameref}

%integrate nomenclature
\input{cfg/inc.ncl-meta.tex}

% depth of contents
\setcounter{secnumdepth}{5}
\setcounter{tocdepth}{5}

% Hyperlinks
\usepackage[breaklinks=true]{hyperref}
\hypersetup{bookmarks=true,breaklinks=true,colorlinks=true,citecolor=blue,draft=false}
\newcommand{\lnka}[1]{\href{#1}{\texttt{#1}}}
\newcommand{\lnkb}[2]{\href{#1}{\texttt{#1} (RDL: #2)}}

\usepackage{amssymb}

\usepackage{wasysym}
\usepackage{harmony}
\usepackage{musicography}
\usepackage{abc}
\usepackage{musixtex}
\usepackage{graphicx,color}
\usepackage{tabulary}
\usepackage{longtable}
\usepackage{threeparttablex}

% package for macking tables with broken lines
\usepackage{multirow}
\usepackage{ulem}
\usepackage{picins}

\usepackage[dvipsnames,table,xcdraw]{xcolor}
\usepackage{pgf}
\usepackage{tikz}
\usetikzlibrary{arrows}
\usetikzlibrary{shapes,snakes}

%\usepackage{bigfoot}
\usepackage{verbatimbox}

% Unfortunately musixtex still uses outdated commands for
% establishing its own \bar command. Hence for enabling
% the use of musixtex we must 'redefine' these outdated commands:
\makeatletter
\DeclareOldFontCommand{\rm}{\normalfont\rmfamily}{\mathrm}
\DeclareOldFontCommand{\sf}{\normalfont\sffamily}{\mathsf}
\DeclareOldFontCommand{\tt}{\normalfont\ttfamily}{\mathtt}
\DeclareOldFontCommand{\bf}{\normalfont\bfseries}{\mathbf}
\DeclareOldFontCommand{\it}{\normalfont\itshape}{\mathit}
\DeclareOldFontCommand{\sl}{\normalfont\slshape}{\@nomath\sl}
\DeclareOldFontCommand{\sc}{\normalfont\scshape}{\@nomath\sc}
\makeatother

\newcommand{\cadlab}[2]{Candenza-#1-#2}

\newcommand{\cad}[2]{
  \begin{flushright}\scshape\label{\cadlab{#1}{#2}}Cadenza-#1: #2\end{flushright}
}
\newcommand{\pnotes}[1]{\notes}

\newcommand{\acc}[0]{\textit}
\newcommand{\ra}[0]{$\rightarrow$}


% -- Basic Colors ---
% LilyPond colors:
\newcommand{\lyColor}[0]{RoyalBlue}
\newcommand{\lyBgColor}[0]{RoyalBlue!10}
% MusicXML colors:
\newcommand{\mxmlColor}[0]{OliveGreen}
\newcommand{\mxmlBgColor}[0]{OliveGreen!10}
% ABC colors:
\newcommand{\abcColor}[0]{Plum}
\newcommand{\abcBgColor}[0]{Plum!10}

% PMX colors:
\newcommand{\pmxColor}[0]{Orange}
\newcommand{\pmxBgColor}[0]{Orange!10}

% MusiXTeX colors:
\newcommand{\mtexColor}[0]{Magenta}
\newcommand{\mtexBgColor}[0]{Magenta!10}

% Xtra colors
\newcommand{\multiBgColor}[0]{Black!10}
\newcommand{\pdfBgColor}[0]{Gray!10}
\newcommand{\backendBgColor}[0]{Yellow!10}

% -- IO Colors ---

\newcommand{\abcInColor}[0]{\abcBgColor}
\newcommand{\abcOutColor}[0]{\abcColor}

\newcommand{\mxmlInColor}[0]{\mxmlBgColor}
\newcommand{\mxmlOutColor}[0]{\mxmlColor}

\newcommand{\lyInColor}[0]{\lyBgColor}
\newcommand{\lyOutColor}[0]{\lyColor}

\newcommand{\pmxInColor}[0]{\pmxBgColor}
\newcommand{\pmxOutColor}[0]{\pmxColor}

\newcommand{\mtexInColor}[0]{\mtexBgColor}
\newcommand{\mtexOutColor}[0]{\mtexColor}

% -- Node types---

\newcommand{\converter}[6]{
  \node[trapezium stretches body,
  fill=#1, text=#2] (#3) at (#4,#5) { \textbf{\sffamily{#6}} };
}

\newcommand{\backend}[6]{
  \node[trapezium stretches body, align=center, minimum height=0.75cm,
  fill=\backendBgColor, text=#2] (#3) at (#4,#5) { \textbf{#6}\\#1};
}

\newcommand{\file}[6]{
  \node[rectangle,draw,text width=1cm,
  fill=#1, text=#2] (#3) at (#4,#5) {\centering{\textbf{\ttfamily{#6}}}};
}

\newcommand{\frontend}[7]{
  \node[rectangle, align=center, rounded corners, minimum height=0.75cm,
  fill=#1, text=#2] (#3) at (#4,#5) { \textbf{#6}\\#7 };
}





\begin{document}

%% use all entries of the bliography
\nocite{*}

%%-- start(titlepage)
\titlehead{\acc{mycrsf} basierte Geisteswissenschaft}
\subject{mwm.ltx-2.1
}
\title{Musikwissenschaft mit \LaTeX \\
  ~\\
  \includegraphics{./logos/mltx-logo.png}\\
  ~\\
}
\subtitle{Wie man Musikbeispiele mit Open-Source-Tools in seine wissenschaftliche Texte integriert.\\ {\small Eine selbstreferentielle Sichtung und Anleitung}}
\author{Karsten Reincke}

%thanks entry cannot be combined with license footnote
%\thanks{den Autoren von KOMA-Script und denen von Jurabib}

\maketitle
%%-- end(titlepage)

\textit{Dieses Tutorial erläutert, wie man musikalische Beispiele mit Open-Source-Mitteln in \LaTeX-Texte einbettet. Dazu sichtet es Notensatzsysteme, Editoren, Konverter und Tools, die Notentexte erzeugen, verändern und in den \LaTeX-Text integrieren. Und es skizziert, wie man ganze 'Produktionsketten' aus Frontendsystemen, Konvertern und Backendsystemen 'zusammenstöpselt'. Zuletzt entsteht so eine 'Landkarte' verschiedener Wege.% mycsrf License Include Module
%
% (c) Karsten Reincke, Frankfurt a.M. 2012, ff.
%
% This file is licensed under the Creative Commons Attribution 3.0 Germany
% License (http://creativecommons.org/licenses/by/3.0/de/):
% For details see teh file LICENSE in the top directory
%
\footnote{\small Unser Text wird unter den Bedingungen der \textit{Creative Commons Share Alike}-Lizenz (\textit{CC BY-SA 4.0}) ver\-öffentlicht: Sie dürfen das Material -- grob gesagt -- in jedem Format oder Medium vervielfältigen und weiterverbreiten, es remixen, verändern und darauf aufbauen -- und zwar für beliebige Zwecke, sogar kommerzielle --, sofern Sie angemessene Urheber- und Rechteangaben machen, einen Link zur Lizenz beifügen, Ihr abgeleitetes Werk unter derselben Lizenz verbreiten und angeben, ob und wo Sie das Original in welcher Hichsicht geändert haben. Details dazu finden Sie unter $\Rightarrow$ \lnka{https://creativecommons.org/licenses/by-sa/4.0/deed.de}.\newline
Das Recht, diese Arbeit im Rahmen des üblichen wissenschaftlichen Verfahrens zu zitieren, bleibt davon unbenommen: es gibt keine Pflicht, das zitierende Werk unter dieselbe Lizenz zu stellen. Die Bedingung der \textit{an\-ge\-mes\-se\-nen Urheber- und Rechteangaben} erfüllen Sie, indem Sie in ihrem Werk an prominenter Stelle den Text einfügen: {\itshape Abgeleitet vom GitHub-Projekt \lnka{https://github.com/kreincke/musicology.ltx}, das von © 2019ff K. Reincke und Kontributoren unter den Bedingungen der Lizenz \acc{CC-BY-SA} veröffentlicht worden ist }\newline
( {\footnotesize Weil wir selbst diese Arbeit von \textit{mycsrf} abgeleitetet haben, fügen wir beispielhaft den Hinweis hinzu: {\itshape Format abgeleitet von \texttt{mind your Scholar Research Framework} \copyright\ K. Reincke CC BY 3.0 DE \lnka{https://github.com/kreincke/mycsrf} }} ) }
}

% \addsec{CO-Autoren / Kontributoren}
%
% \begin{description}
%   \item[Karsten Reincke] Initiator and main author up to release
% \end{description}


\addchap{Vorwort}


% (c) Karsten Reincke, Frankfurt a.M. 2012, ff.
%
% This text is licensed under the Creative Commons Attribution 3.0 Germany
% License (http://creativecommons.org/licenses/by/3.0/de/): Feel free to share
% (to copy, distribute and transmit) or to remix (to adapt) it, if you respect
% how you must attribute the work in the manner specified by the author(s):
% \newline
% In an internet based reuse please link the reused parts to mycsrf.fodina.de
% and mention the original author Karsten Reincke in a suitable manner. In a
% paper-like reuse please insert a short hint to mycsrf.fodina.de and to the
% original author, Karsten Reincke, into your preface. For normal quotations
% please use the scientific standard to cite
%

Ich selbst hatte mir zu Beginn eines größeren musikwissenschaftlichen Projektes gewünscht, ein solches Tutorial zu haben: \acc{\LaTeX}, \acc{Bib\TeX}\ und \acc{JabRef} waren mir schon vertraut. Meinen 'optimalen' Zitierstil für das Schreiben geisteswissenschaftlicher Arbeiten hatte ich bereits konfiguriert\footnote{\cite[vgl.][\nopage wp]{Reincke2018a}. Mittlerweile ist mit \acc{\LaTeX}, \acc{Bib\LaTeX}, \acc{biber} und \acc{biblatex-dw} ein deutlich modernerer Systemkomplex entstanden, um geisteswissenschaftliche Texte zu erstellen. Diese Alternative habe ich im Repository \lnka{https://github.com/kreincke/proScientia.ltx} analog zu \acc{mycsrf} entwickelt. Nachdem das geschafft war, lag es nahe, auch den hier vorliegenden Text \acc{Musikwissenschaft mit \LaTeX} Werk auf die neuere, bessere Technik zu migrieren. Leider zeigte ein Versuch, dass \acc{Music\TeX} (irgendwie) mit \acc{Bib\LaTeX}\ kollidiert. So ist eine direkte Umstellung von \acc{mwm.ltx} (z.Zt.) nicht möglich. Einer Umstellung des musikwissenschaftlichen Arbeitens  auf die moderne Technik steht jedoch nichts entgegen, wenn man sich auf den zweiten Weg via \LaTeX mit \acc{Lilypond} fokussiert. Dies habe ich in und mit dem Repository \lnka{https://github.com/kreincke/proMusicologica.ltx} ausgearbeitet.} und dokumentiert\footnote{\cite[vgl][2ff]{Reincke2018b}.}. Unklar war mir 'nur', wie man Notentexte und musikalische Analysen in \LaTeX-Texte einbindet.

Das sollte eigentlich nicht kompliziert sein. Man bräuchte dazu doch nur ein No\-ta\-tions\-system, das Notentext erfasst und das -- als Teil des \LaTeX-Quelltextes -- das Notenbild in das eigentlich Werk hinein generiert. Leider schwiegen sich meine sehr guten, einschlägigen \LaTeX-Bücher dazu aus\footcite[vgl.][vi ff, insbesondere 905 u. 909: das umfangreiche Register erwähnt weder Musik im allgemeinen noch LilyPond oder MusiX\TeX\ im Besonderen]{Voss2012a}, selbst wenn sie auch Randbereiche behandelten\footcite[vgl.][vii ff, insbesondere 1080 u. 1087: auch dieses umfangreiche Register erwähnt weder Musik im allgemeinen noch LilyPond oder MusiX\TeX\ im Besonderen.]{MitGoo2005a}. Die entsprechende Internetrecherche überrollte mich dagegen: so viele Notationssysteme und Tools, aber kein systematischer Überblick.\footnote{Rühmlich die Ausnahme von \cite[][\nopage wp.]{Thoma2018a}. Allerdings ging sie nicht in die Tiefe, die ich benötigte.}

Wollte ich meine Arbeit also nicht gefährden und Sackgassen vermeiden, musste ich die gegebenen Möglichkeiten zuerst sichten. Andernfalls wäre ich Gefahr gelaufen, zuletzt doch 'aufs falsche Pferd gesetzt' zu haben. Was ich brauchte war eine technisch fundierte 'Landkarte' der Methoden und Tools, die mir den besten Weg weisen konnte. Und so traf mich -- wieder einmal -- die Erkenntnis:

\begin{quote}\textit{Was man im Internet nicht findet, muss man selbst hineinstellen - am Besten unter einer Lizenz, die jedem die freie Wieder- und Weiterverwendbarkeit garantiert.} \end{quote}

\begin{flushright}
Ehringshausen OT Katzenfurt, \today: Karsten Reincke
\end{flushright}
 % this is only inserted to eject fault messages in texlipse
%\bibliography{../bib/literature}


\footnotesize
\tableofcontents

\normalsize
\setcounter{chapter}{-1}
\chapter{Der Kontext}

\input{snippets/inc.anliegen.tex}

\input{snippets/inc.kriterien.tex}

\chapter{Leichtgewichtige Teillösungen}

Selbstverständlich wird sich zuletzt herausstellen, dass adäquate Systeme zur
Erfassung von Musik eine gewisse Komplexität mitbringen. Das liegt in der Natur
der Sache: Musik ist zweidimensional.\footnote{Bei der Notation von Musik hatten
wir gesagt, dass sie zweidimensional oder eindimensional codiert sein könne.
Als Abfolge(sic!) von Klängen(sic!) ist sie selbst aber immer (mindestens)
zweidimensional.} Das hat einige Autoren nicht entmutigt, wenigstens für
Teilaufgaben einfachere Lösungen zu entwickeln, die im freien Textsatzsystem
\LaTeX\footnote{Bzgl. der Zusammenhänge von \TeX, \LaTeX, pdf\LaTeX\ und
Lua\LaTeX\ vlg. etwa \cite[][7ff]{Voss2018a} oder \cite[][8ff]{Voss2012a}.
Daraus ergibt sich mittelbar auch der Open-Source-Status von \LaTeX: \TeX\ und
\LaTeX\ sind zunächst einmal 'nur' eine Metasprache. Um aus Dateien, die in
dieser Auszeichnungssprache formuliert sind, typographisch gestaltete \acc{DVI}-
oder \acc{PDF}-Dateien zu erzeugen, bedarf des bestimmter Programme, nämlich
\texttt{pdflatex}, \texttt{lualatex}\ etc. Außerdem hat sich um \LaTeX\ herum ein
Biotop von Paketen entwickelt. Das sind fertig vorbereitete \TeX- oder
\LaTeX-Dateien, die man in seine eigenen Dokumente einbindet und also mitnutzt.
Solche Pakete stehen -- mit den Kernprogrammen gebündelt -- als komplette
\TeX-Distributionen bereit. Für den europäischen Raum existieren z.Zt. zwei
wesentliche Distributionen, \acc{{\TeX}live} und \acc{MiK{\TeX}}. Diese findet
man im \acc{Comprehensive TeX Archive Network} ($\rightarrow$
\href{https://ctan.org/}{https://ctan.org/}). Der unbedarfte Anwender tut aber
gut daran, sich vorbereitete Installationspakete von sekundären Distributoren zu
holen. \acc{Linux}-Distributionen liefern diese i.d.R. mit. Ob man sein
\LaTeX-System lizenzkonform nutzt, entscheidet also die Lizensierung der
Kernprogramme und die Lizenzen der eingebundenen Pakete. Diese müssen nicht
unter derselben Lizenz veröffentlicht sein. Glücklicherweise werden beide -- die
Kernprogramme und die Erweiterungen -- als \acc{CTAN}-Pakete gehostet und unter
den Bedingungen verschiedener FOSS-Lizenzen zum Download angeboten. So ist etwa
pdf\LaTeX\ unter der GPL lizenzisiert ($\rightarrow$
\href{https://ctan.org/pkg/pdftex} {https://ctan.org/pkg/pdftex}). Ganz
allgemein gesagt, darf man alles, was zu \LaTeX\ gehört, als freie Software
nutzen. Wir werden bei den Paketen jedoch kurz angeben, unter welcher Lizenz sie
genau stehen.} genutzt werden können.
Diesen werden wir zuerst nachgehen. Behalten wir aber im Hinterkopf:
\acc{Komplexität ist wie Wasser: man kann es nicht
komprimieren.}\footnote{Dieses wunderbare Bonmot stammt nicht von mir. Und
bedauerlicherweise weiß ich auch nicht (mehr) von wem. Ich meine es etwa Anfang
der 2010er Jahre in einer Keynote bei der Deutschen Telekom in Darmstadt gehört
zu haben. Ich verbeuge mich also dankbar vor dem mir leider unbekannten Autor.}


\input{snippets/inc.sonderzeichen.tex}

\input{snippets/inc.wasysym.tex}

% mycsrf 'for beeing included' snippet template
%
% (c) Karsten Reincke, Frankfurt a.M. 2012, ff.
%
% This text is licensed under the Creative Commons Attribution 3.0 Germany
% License (http://creativecommons.org/licenses/by/3.0/de/): Feel free to share
% (to copy, distribute and transmit) or to remix (to adapt) it, if you respect
% how you must attribute the work in the manner specified by the author(s):
% \newline
% In an internet based reuse please link the reused parts to mycsrf.fodina.de
% and mention the original author Karsten Reincke in a suitable manner. In a
% paper-like reuse please insert a short hint to mycsrf.fodina.de and to the
% original author, Karsten Reincke, into your preface. For normal quotations
% please use the scientific standard to cite
%


%% use all entries of the bibliography

\section{Noch mehr Sonderzeichen: musicography ($\bigstar\bigstar$)}

Das Zusatzpaket \acc{musicography}\footnote{\cite[vgl.][\nopage
wp.]{CtanMusicography2018a}. Die Paketbeschreibung gibt an, dass
\acc{musicography} unter der \acc{LaTeX\ Project Public Li­cense} veröffentlicht
wird. Das ist eine von der \acc{OSI} anerkannte Open-Source-Lizenz
($\rightarrow$ \href{https://opensource.org/licenses/LPPL-1.3c}
{https://opensource.org/licenses/LPPL-1.3c}).} vereinigt und erweitert die
bisher erwähnten Möglichkeiten. Außerdem sagen manche Fürsprecher, dass es -- im
Gegensatz zu anderen Paketen -- auch mit \textit{pdflatex} druckfähige
PDF-Dateien erzeuge.\footnote{\cite[Vgl. dazu etwa][1]{Cashner2018a}. Persönlich
sind uns bei der PDF-Generierung solche Irritationen mit anderen Paketen nicht
begegnet. \acc{mycsrf} nutzt auch \textit{pdflatex}. Und unter Ubuntu 18.04
werden dabei alle Fonts eingebunden und können alles ausdrucken, was auch am
Bildschirm sichtbar ist: so auch bei dem Beispiel \textit{musicology.de}.}
Selbstverständlich muss dieses Paket seiner \LaTeX-Natur nach ebenfalls in die
Präambel des \LaTeX-Dokumentes eingebunden werden (\texttt{\small
\textbackslash{usepackage\{musicography\}}}), um entsprechende Befehle nutzen zu
können.

Anschließend erlaubt das Paket, Vorzeichen \{
\musFlat \ (= \texttt{\small \textbackslash{musFlat}}),
\musSharp \ (= \texttt{\small \textbackslash{musSharp}}),
\musNatural \ (= \texttt{\small \textbackslash{musNatural}}),
\musDoubleFlat \ (= \texttt{\small \textbackslash{musDoubleFlat}}),
\musDoubleSharp \ (= \texttt{\small \textbackslash{musDoubleSharp}})
\}, Notensymbole \{
\musWhole \ (= \texttt{\small \textbackslash{musWhole}}),
\musHalf \ (= \texttt{\small \textbackslash{musHalf}}),
\musQuarter \ (= \texttt{\small \textbackslash{musQuarter}}),
\musEighth \ (= \texttt{\small \textbackslash{musEighth}}),
\musSixteenth \ (= \texttt{\small \textbackslash{musSixteenth}}),
\musHalfDotted \ (= \texttt{\small \textbackslash{musHalfDotted}}),
\musQuarterDotted \ (= \texttt{\small \textbackslash{musQuarterDotted}})
\ldots
\}
und Metren wie \meterCutC \ (= \texttt{\small \textbackslash{meterCutC}})
in den Fließtext einzuarbeiten.

Gleichwohl gibt es 'Ungereimtheiten', die den Gebrauch erschwert: Zunächst muss man das Paket \textit{MusiX\TeX}, sofern man es zusammen mit \textit{musicology} verwenden will, nach \textit{musicology} in die Präambel einbinden. Andernfalls 'meckert' \textit{musicology}, dass der Befehl \texttt{meterC} schon definiert sei. Bindet man \textit{MusiX\TeX} tatsächlich nach \textit{musicology} ein, kann \textit{MusiX\TeX} zwar den von \textit{musicology} eingeführten Befehl problemlos redefinieren. Nur passen die Symbole dann -- wie hier zu sehen -- von der Größe her nicht mehr zu einander: \textit{musicology} $\rightarrow$ \meterCutC : \meterC $\leftarrow$ \textit{MusiX\TeX}.

Wenn man dieses Paket verwenden will, ist also eine gewisse Vorsicht angesagt.


% this is only inserted to eject fault messages in texlipse
%\bibliography{../bib/literature}


% mycsrf 'for beeing included' snippet template
%
% (c) Karsten Reincke, Frankfurt a.M. 2012, ff.
%
% This text is licensed under the Creative Commons Attribution 3.0 Germany
% License (http://creativecommons.org/licenses/by/3.0/de/): Feel free to share
% (to copy, distribute and transmit) or to remix (to adapt) it, if you respect
% how you must attribute the work in the manner specified by the author(s):
% \newline
% In an internet based reuse please link the reused parts to mycsrf.fodina.de
% and mention the original author Karsten Reincke in a suitable manner. In a
% paper-like reuse please insert a short hint to mycsrf.fodina.de and to the
% original author, Karsten Reincke, into your preface. For normal quotations
% please use the scientific standard to cite
%


%% use all entries of the bibliography


\section{Ganz viele Sonderzeichen: harmony ($\bigstar\bigstar\bigstar\bigstar\bigstar$)}
\label{Harmony}
Das Paket \textit{harmony}\footnote{\cite[vgl.][\nopage wp.]{CtanHarmony2018a}.
Die Paketbeschreibung gibt an, dass \acc{harmony} unter der \acc{LaTeX\ Project
Public Li­cense} veröffentlicht wird. Das ist eine von der \acc{OSI} anerkannte
Open-Source-Lizenz ($\rightarrow$ \href{https://opensource.org/licenses/LPPL-1.3c}
{https://opensource.org/licenses/LPPL-1.3c}).} ist ein Highlight, bietet es für
den Gebrauch innerhalb einer Textzeile doch ein besonders ausgefeiltes, einfach
anzuwendendes System von Musikzeichen an. Als \LaTeX-Paket muss es natürlich
ebenfalls zuerst mittels eines Befehls in die Präambel eingebunden werden
(\small \texttt{\textbackslash{usepackage\{harmony\}}}), bevor es Zeichen für die beiden
Bereiche \textit{Rhythmik} und \textit{Harmonieanalyse}
bereitstellt\footcite[Für einen vollen Überblick über den Zeichenvorrat und die
Kombinationsmöglichkeiten vgl.][4ff]{WegWeg2007a}:

\subsection{\small Rhythmik}

Zunächst enthält es gute aufgelöste Kodierungen für Taktarten \{
\Takt{3}{4} \ (= \texttt{\small \textbackslash{Takt}\{3\}\{4\}}),
\Takt{4}{4} \ (= \texttt{\small \textbackslash{Takt}\{4\}\{4\}}),
\ldots,
\Takt{c}{0} \ (= \texttt{\small \textbackslash{Takt}\{c\}\{0\}}),
\Takt{c}{1} \ (= \texttt{\small \textbackslash{Takt}\{c\}\{1\}})
\}.
Dann offeriert es nicht nur einfache Notenlängen \{
\Ganz \ (= \texttt{\small \textbackslash{Ganz}}),
\Halb \ (= \texttt{\small \textbackslash{Halb}}),
\Vier \ (= \texttt{\small \textbackslash{Vier}}),
\Acht \ (= \texttt{\small \textbackslash{Acht}}),
\Sech \ (= \texttt{\small \textbackslash{Sech}}),
\Zwdr \ (= \texttt{\small \textbackslash{Zwdr}}),
\}  -- \ die sogar punktiert werden können
\{
\Halb\Pu \ (= \texttt{\small \textbackslash{Halb}\textbackslash{Pu}}),
\Vier\Pu \ (= \texttt{\small \textbackslash{Vier}\textbackslash{Pu}}),
\ldots
\}
-- , und die dazu passenden Pausensymbole
\{
\GaPa \ (= \texttt{\small \textbackslash{GaPa}}),
\HaPa \ (= \texttt{\small \textbackslash{HaPa}}),
\ViPa \ (= \texttt{\small \textbackslash{ViPa}}),
\AcPa \ (= \texttt{\small \textbackslash{AcPa}}),
\SePa \ (= \texttt{\small \textbackslash{SePa}}),
\},
sondern es stellt auch Symbole für Noten mit waagerechtem Balken zu Verfügung
\{
\AchtBL \ (= \texttt{\small \textbackslash{AchtBL}}),
\SechBL \ (= \texttt{\small \textbackslash{SechBL}}),
\Vier\AchtBL \ (= \texttt{\small \textbackslash{Vier} \textbackslash{AchtBL} }),
\Vier\SechBL \ (= \texttt{\small \textbackslash{Vier} \textbackslash{SechBL} }),
\ldots
\},
sodass die einzelnen Elemente -- als Syntagmen aneinandergereiht -- zu ganzen
Rhythmusketten kombiniert werden können:
\Takt{c}{0} \Vier \ \Vier\AchtBL \ \Vier\Pu \ \Acht \ $|$ \AchtBR\Pu \SechBl \
\AchtBR\kern-0.15em\SechBR\Vier \ \SechBr\Vier\SechBl \ $|$ \ -- eine wirklich
ausgefuchste Lösung.

\subsection{\small Harmonik}

Einen ähnlich geschickten Ansatz bietet das Paket \textit{harmony} da, wo es
Symbole erzeugt, die -- der Funktions- und Stufentheorie entsprechend --
harmonische Zusammenhänge repräsentieren.

Den Kern bildet die allgemeine Tupelkonstruktion \texttt{\small
\textbackslash{HH.X.u.v.w.z.}}, mit der einfache und komplexe Aspekte
dargestellt werden können\footcite[vgl. dazu][2ff]{WegWeg2007a}: Was zwischen
den ersten beiden Punkten erscheint (X), erscheint als Hauptzeichen; dann folgt
das, was darunter erscheinen soll (u) und schließlich das, was rechts oben
daneben angezeigt werden soll, und zwar von oben nach unten (v w z):

\begin{center}
\HH.X.u.v.w.z.
\end{center}

Der Witz ist nun, dass man nahezu beliebige Latexkonstrukte zwischen den Punkten
einsetzen kann. Und wenn man nichts zwischen den Punkten einträgt, bleibt der
Platz leer. So können die einfachen Symbole der Funktionstheorie \{ \HH.T.....
\ (= \texttt{\small \textbackslash{HH.T.....}}), \HH.Tp.....  \ (=
\texttt{\small \textbackslash{HH.Tp.....}}), \HH.S.....  \ (= \texttt{\small
\textbackslash{HH.S.....}}), \HH.D.....  \ (= \texttt{\small
\textbackslash{HH.D.....}}) \} mit derselben Technik erzeugt werden, wie die
komplexeren \{ \HH.D.3.9.7..  \ (= \texttt{\small
\textbackslash{HH.D.3.9.7..}}), \HH.T..9$\flat\rightarrow$8.7..  (=
\verb|\HH.T..9$\flat\rightarrow$8.7..|).
Die speziellen Zeichen der Funktionstheorie, die nicht so einfach aus dem Fundus
normaler Fonts gebildet werden können, stellt  \textit{harmony} gesondert zur
Verfügung:
\{ \Dohne  \ (= \texttt{\small \textbackslash{Dohne}}), \DD \ (= \texttt{\small
\textbackslash{DD}}), \DS  \ (= \texttt{\small \textbackslash{DS}}) \}.
Selbstverständlich können diese ebenfalls in das allgemeine Tupelkonstrukt
eingebettet werden\footcite[Vgl. dazu][6]{WegWeg2007a}:
\begin{center}
 \texttt{\textbackslash{HH}.\textbackslash{DD}.5\textbackslash{VM}.7...}
 $\rightarrow$ \HH.\DD.5\VM.7...
\end{center}

Zudem erlaubt es die Flexibilität des Grundkonstruktes (in gewissen Grenzen),
Symbole für die Stufentheorie und den Generalbass zu erzeugen:

\begin{center}
\HH.I..\texttt{(5)}.\texttt{(3)}.. \
\HH.III..\texttt{ 6 }.\texttt{(3)}.. \
\HH.V..\texttt{ 6}.\texttt{ 4}.. \
\HH.I..\texttt{(5)}.\texttt{(3)}.. \
\HH.I..\texttt{(5)}.\texttt{ 3$\flat$ }.. \
\HH.III..\texttt{ 6 }.\texttt{ 3$\flat$}.. \
\HH.III..\texttt{ 6 }.\texttt{ 5 }.\texttt{(3)}. \
\HH.III..\texttt{ 6$\flat$}.\texttt{ 3$\flat$}.. \
\end{center}

An diesem Beispiel erkennt man außerdem, dass sich die
\textit{harmony}-Konstrukte (hier \texttt{\textbackslash{texttt}} und
\texttt{\$\textbackslash{flat}\$} ) gut mit anderen \LaTeX-Elementen kombinieren
lassen:
\begin{verbatim}
\begin{center}
\HH.I..\texttt{(5)}.\texttt{(3)}.. \
\HH.III..\texttt{ 6 }.\texttt{(3)}.. \
\HH.V..\texttt{ 6}.\texttt{ 4}.. \
\HH.I..\texttt{(5)}.\texttt{(3)}.. \
\HH.I..\texttt{(5)}.\texttt{ 3$\flat$ }.. \
\HH.III..\texttt{ 6 }.\texttt{ 3$\flat$}.. \
\HH.III..\texttt{ 6 }.\texttt{ 5 }.\texttt{(3)}. \
\HH.III..\texttt{ 6$\flat$}.\texttt{ 3$\flat$}.. \
\end{center}
\end{verbatim}

Später werden wir zeigen, dass sich die \textit{harmony}-Elemente ihrerseits
auch gut in MusiX\TeX-Syntagmen einbetten lassen. Insofern haben die
Programmierer von \textit{harmony} der Community ein mächtiges Werzeug zur
Verfügung gestellt.\footnote{Trotzdem wollen auch wir wenigstens darauf
hinweisen, dass \textit{harmony}-Konstrukte auf die Einbettung in einen
Fließtext mit 12 Pt. ausgelegt sind. Bei kleineren Größen von 11PT abwärts
werden die Zeilenabstände gedehnt, es entsteht ein leicht unruhigeres
Druckbild. (\cite[Vgl. dazu][2]{WegWeg2007a}.) Das zu beklagen, wäre allerdings
ein Jammern auf sehr hohem Niveau.}


% this is only inserted to eject fault messages in texlipse
%\bibliography{../bib/literature}


\chapter{Backends: Komplexe Notationssysteme}

\input{snippets/inc.abc.tex}

\input{snippets/inc.musixtex.tex}

\input{snippets/inc.pmx.tex}

\input{snippets/inc.lilypond.tex}

\input{snippets/inc.graphics.tex}

\input{snippets/inc.mup.tex}
\input{snippets/inc.texmuse.tex}

\chapter{Frontends: die (graphische) Eingabe}

Abgesehen von der letzten Variante, werden bei den hier diskutierten Methoden
die Noten zuletzt textbasiert codiert: um zu komponieren oder zu arrangieren
benötigt man einen Texteditor. Dessen Output wird \acc{ABC}, \acc{LilyPond},
\acc{MusiX\TeX}\ oder \acc{PMX} als die Backendsysteme übergeben, die daraus
den eigentlichen Notentext erzeugen.

Die Arbeit mit reinen Texteditoren ist nicht das, was man unter einer
musikerfreundlichen Arbeitsumgebung verstehen würde. Die Sprache der Musiker
sind Noten, keine mehr oder minder kryptischen Syntagmen. Und so gibt es denn
auch eine Reihe von graphischen Programmen, bei denen der Musiker entweder Noten
'schreibt', nicht Text, oder bei denen seine Texteingaben wenigstens unmittelbar
visualisiert werden, ohne dass er zuvor selbst das Backend aufrufen müssten.
Erstere bezeichnen wir als graphische oder visuelle Editoren, letztere als
semi-graphische.\footnote{Für den unbedarften Nutzer sieht es manchmal so aus,
als sei die Erzeugung des Notentextes integrierter Bestandteil des Frontends.
Wäre dem so, schiene der Begriff 'Editor' irgendwie ungerecht. Tatsächlich
verküpfen solche 'integrierten Entwicklungsumgebungen' meist aber den Editor
'nur' sehr geschickt mit der eigentlichen 'Notengenerierungsmaschine', dem
Backend. Deshalb ist es in unserem Kontext sehr wohl sinnvoll, von Backends und
Frontends zu sprechen und unter letzteren im Wesentlichen Editoren zu verstehen,
die die eingebenen Daten in verschiedenen Formaten abzuspeichern vermögen, --
selbst wenn sie sich selbst eher als IDE, als \acc{Integrated Development
Environment} verstehen.}

Damit darf man fragen, ob und wie man Opensource-Notensatzprogramme dazu nutzen
kann, den zuletzt im \LaTeX-Dokument benötigten Code aus dem abzuleiten, was man
in und mit diesen Editoren eingegeben. Diese Frage ist im Einzelfall -- und
nicht nur für Opensourceprogramme -- einfach zu beantworten:

\begin{itemize}
\item Zum einen muss man überprüfen, ob es der Editor erlaubt, den Inhalt in
einem der benötigten Formate zu speichern.
\item Und zum anderen muss man testen, ob der Editor den Inhalt auch
hinreichend 'verlustfrei' exportiert.
\end{itemize}

Dem werden wir nachgehen. Allerdings wollen wir noch etwas vorausschicken:

Frontends von Notensatzsystemen nutzen meist ein natives Dateiformat und
exportieren ihren Inhalt ggfls. in Fremdformate. Sie agieren implizit als
Konverter. Daneben gibt es noch eine Reihe von expliziten Konvertern, die eine
Datei in dem einen Format einlesen und in einem anderen wieder abspeichern.

Sollte also ein Notensatzprogramm von sich aus das eine oder andere der in
unserem Kontext benötigten Formate nicht oder nur unzureichend exportieren,
bliebe immer noch die Möglichkeit, einen eigenständigen Konverter
zwischenzuschalten. Zu wissen, welche expliziten Konverter es (in welcher
Qualität) gibt, könnte mithin die Anzahl der gut zu nutzenden Notensatzsysteme
erhöhen.\footnote{Wollte man diesem Gedanken in letzter Konsequenz nachgehen,
müsste man zunächst eine Liste aller Notationsformate erstellen und dann nach
Konvertern suchen, die diese 'anderen' auf die Formate \textit{ABC},
\textit{Musix\TeX}, \textit{PMX} oder \textit{LilyPond} abbilden. Das kann beliebig
komplex werden. Wir konzentrieren uns hier auf die freie Software. Formate, die
proprietäre Programme verwenden, geraten also gar nicht erst in unseren Blick.
Von daher würde eine Übersicht der Konverter, die wir erstellen, niemals alle
Wege abdecken. Insofern erlauben wir uns, die Konverter 'nur' kursorisch zu
sichten.}


\input{snippets/inc.converter.tex}

\input{snippets/inc.editoren.tex}

\input{snippets/inc.exkurs-midi.tex}

\section{Editoren II}

\input{snippets/inc.abcj.tex}

\input{snippets/inc.aria.tex}

\input{snippets/inc.audimus.tex}
\newpage
% mycsrf 'for beeing included' snippet template
%
% (c) Karsten Reincke, Frankfurt a.M. 2012, ff.
%
% This text is licensed under the Creative Commons Attribution 3.0 Germany
% License (http://creativecommons.org/licenses/by/3.0/de/): Feel free to share
% (to copy, distribute and transmit) or to remix (to adapt) it, if you respect
% how you must attribute the work in the manner specified by the author(s):
% \newline
% In an internet based reuse please link the reused parts to mycsrf.fodina.de
% and mention the original author Karsten Reincke in a suitable manner. In a
% paper-like reuse please insert a short hint to mycsrf.fodina.de and to the
% original author, Karsten Reincke, into your preface. For normal quotations
% please use the scientific standard to cite
%


%% use all entries of the bibliography

\subsection{Brahms (-)}

\parpic(1cm,1cm)[r][t]{\includegraphics[width=1cm]{logos/brahms-300dpi.png}}
\label{Brahms}\acc{Wikipedia} zählt \acc{Brahms} zu den \enquote{Sequenzern},
die \enquote{[\ldots] neben ihrem Hauptanwendungsfeld der Audio- und
MIDI-Bearbeitung auch Notensatzfunktionalitäten
(beinhalten)}.\footcite[vgl.][\nopage wp.]{WpedNotensatz2019a} Ein langer und älterer Artikel beschreibt seine Arbeitsweise genauer.\footnote{\cite[vgl.][\nopage wp.]{Wuerthner2002a} G. Spahlinger hat uns dankenswerterweise auf diese Quelle aufmerksam gemacht, sodass wir sie nun ab der Version \acc{mwm.ltx-2.1} auch auswerten konnten. Geändert hat sie an unserem Befund nichts. } Wir sind trotzdem geneigt, von einem mittlerweile 'geghosteten' Program zu sprechen:

Bis 2019 war noch eine Sichtung von Musiksoftware aus dem Jahr 2009 zugänglich, die das Program \acc{Brahms} erwähnte und mit einem Sourceforgeprojekt verlinkte.\footnote{\cite[vgl.][\nopage wp.]{Callon2009a}. Dieser Link ist allerdings in 2022 nicht mehr erreichbar.} Eine aktuelle Sichtung auf Wikipedia listet \acc{Brahms} zwar noch unter der Kategorie \enquote{Sequenzer mit Notensatzfunktion}, kann aber ein Repository oder eine Homepage nicht mehr angeben.\footcite[vgl.][\nopage wp.]{WpedNotensatz2019a}

Die Sichtung aus dem Jahr 2009 und der Artikel über \acc{Brahms} aus dem Jahr 2002 verweisen als Bezugspunkt auf dasselbe Projekt auf Sourceforge, dessen Homepage dann -- beruhigender- und letztlich irreführenderweise -- ein Logo mit Noten nutzt.\footcite[vgl.][\nopage
wp.]{Brahms2013a} Dennoch hat (dieses) \acc{Brahms} (von sich aus) nichts mit
Musik zu tun: es sei ein \enquote{[\ldots] Modular Execution Framework (MEF) for
executing integrated systems built from component software processes}, ein
\enquote{SystemML-ready execution client}.\footcite[vgl.][\nopage
wp.]{Brahms2013b} Mittlerweile gibt es dazu ein neueres Github-Repository, das
per 'Fork' aus dem Sourceforgeprojekt entstanden ist. Und unter Github
beschreibt sich \acc{Brahms} -- direkt und ganz ohne Notenlogo -- als
\enquote{simulation execution engine}.\footcite[vgl.][\nopage wp.]{Brahms2018a}

Tatsächlich ist (dieses) \acc{Brahms} kein Sequencer mit Notensatzfunktion. Und
wenn es solch ein Programm früher einmal gegeben hat, dann
sind die Repositories durcheinander geraten. Oder aber das richtige Notensatz-
und Sequencerprogramm \acc{Brahms} wird so versteckt gepflegt, dass auch der
unbedarfte Musikwissenschaftler es nicht einfach und schnell genug finden wird.
So läuft alles auf dasselbe hinaus: heute ist \acc{Brahms} in dieser Funktion
nicht nutzbar.


% this is only inserted to eject fault messages in texlipse
%\bibliography{../bib/literature}


\input{snippets/inc.canorus.tex}

\input{snippets/inc.denemo.tex}

\input{snippets/inc.easyabc.tex}

\input{snippets/inc.elysium.tex}

\input{snippets/inc.freeclef.tex}

\input{snippets/inc.frescobaldi.tex}

\input{snippets/inc.jniz.tex}

\input{snippets/inc.laborejo.tex}

\input{snippets/inc.musedit.tex}

\input{snippets/inc.musescore.tex}

\input{snippets/inc.mux2d.tex}

\input{snippets/inc.noteedit.tex}

\input{snippets/inc.nted.tex}

\input{snippets/inc.ptolemaic.tex}

\input{snippets/inc.rosegarden.tex}

\input{snippets/inc.converterii.tex}

\chapter{Vom Frontend nach \LaTeX: die Toolketten}
\input{snippets/inc.toolchain-survey.tex}
\input{snippets/inc.toolchain-evaluation.tex}

% mycsrf 'for beeing included' snippet template
%
% (c) Karsten Reincke, Frankfurt a.M. 2012, ff.
%
% This text is licensed under the Creative Commons Attribution 3.0 Germany
% License (http://creativecommons.org/licenses/by/3.0/de/): Feel free to share
% (to copy, distribute and transmit) or to remix (to adapt) it, if you respect
% how you must attribute the work in the manner specified by the author(s):
% \newline
% In an internet based reuse please link the reused parts to mycsrf.fodina.de
% and mention the original author Karsten Reincke in a suitable manner. In a
% paper-like reuse please insert a short hint to mycsrf.fodina.de and to the
% original author, Karsten Reincke, into your preface. For normal quotations
% please use the scientific standard to cite
%




\chapter{Fazit}

Damit haben wir das Ende unser Untersuchung erreicht. Fassen wir das Unterfangen
zusammen:

Aus Anlass einer anstehenden, größeren musikwissenschaftlichen Arbeit mussten
wir in \LaTeX-Texte Notenbeispiele einbetten können, und zwar samt der
ausgefeilten Symbolkomplexe, wie sie in der funktionalen Harmonieanalyse üblich
sind. Eine einfache Anleitung dazu gab bisher nicht. Die Suche im Netz verwies
stattdessen auf eine Fülle von Tools und Techniken, deren Nutzen und Nutzbarkeit
und Kombinierbarkeit unklar blieb. So haben wir dieses Feld ausgeleuchtet, um
den von der Qualität besten und von der Handlichkeit einfachsten Weg zu finden:

Notenbeispiele können auf verschiedenen Wegen in \LaTeX-Texten eingebettet
werden. Wir haben drei Backendsystem gefunden, nämlich
\acc{\LaTeX\,+\,ABC}, \acc{\LaTeX\,+\,MusiX\TeX} und
\acc{\LaTeX\,+\,LilyPond}.

Letztlich kann man aus den Tools in und um \acc{\LaTeX\,+\,ABC} kein akzeptables
Editiersystem für Musikwissenschaftler zusammenstellen. Wenn man unbedingt will,
steht der \acc{EasyABC}-Editor und sein Export nach \acc{musicxml} bereit, um
von dort aus über den Konverter \acc{musicxml2ly} in die \acc{LilyPond}-Welt
hinüberzuleiten. Dass der Nusikwissenschaftler dann zusätzlich immer auch
\acc{LilyPond} können muss, ergibt sich aus der Tatsache, dass die wirklich
komplexen Harmonyanalysesymbole erst über die \acc{LilyPond}-Zusatzbibliothek
\acc{harmonyli.ly} in das Notenbild eingebracht werden können. Ob und in wie
weit es Sinn macht, dafür zwei Repräsensationssprachen -- also \acc{ABC} und
\acc{LilyPond} -- zu lernen, möge jeder für sich selbst entscheiden.


Deutlich klarer sieht die Lage bei \acc{\LaTeX\,+\,MusiX\TeX} und
\acc{\LaTeX\,+\,LilyPond} aus:

\acc{\LaTeX\,+\,MusiX\TeX} -- im Verbund mit dem \LaTeX-Tool \acc{harmony} --
und \acc{\LaTeX\,+\,LilyPond} -- in Kombination mit der \acc{LilyPond}
Bibliothek \acc{harmonyli.ly} -- bieten von sich aus die Option, hinreichend
ausdrucksstarke Harmonieanalysesymbole in die Notenbeispiele zu integrieren.
Das Druckergebnis ist exzellent und ganz auf der Höhe, die ein \LaTeX-Nutzer
erwartet.

Der \textbf{Vorteil} des \acc{\LaTeX\,+\,\textbf{MusiX\TeX}}-Ansatzes liegt in
der bruchlosen Integration in das gewohnte \LaTeX-Handling und in der Tatsache,
dass \LaTeX-Syntagmen in MusiX\TeX-Bereichen verwendeten werden können.

Der erste \textbf{Nachteil} der \acc{\LaTeX\,+\,\textbf{MusiX\TeX}}-Methode liegt in
der Komplexität und Unhandlichkeit der Auszeichnungssprache \acc{MusiX\TeX}.
Leider gibt es kein graphisches oder semi-graphisches Frontend für dieses
Backend. Die Idee, ein existierendes Frontend über einen Konverter dafür nutzbar
zu machen, scheitert entweder daran, dass es keinen entsprechenden Konverter
gibt (\acc{ly2musictex}) oder dass die existierenden Konverter nicht adäquat
arbeiten (\acc{abc2ly} resp. \acc{abc2mtex}).

Der zweite \textbf{Nachteil} der \acc{\LaTeX\,+\,\textbf{MusiX\TeX}}-Methode besteht darin, dass Music\TeX\ und Bib\LaTeX\ bzw. Biber -- wenigstens zur Zeit\footnote{Stand 2022-03-13} -- zu kollidieren scheinen. Damit verschließt sich Music\TeX dagegen, in der moderneren Variante von \acc{Musikwissenschaft mit \LaTeX}\footcite[vgl.][\nopage wp.]{Reincke2022a} eingesetzt zu werden.

Der \textbf{Nachteil} des \acc{\LaTeX\,+\,\textbf{LilyPond}}-Ansatzes besteht
dagegen darin, dass die Notenbeispiele nicht nativ in den \LaTeX-Text
eingebettet werden, sondern 'nur' als vorab erzeugte Graphik. Damit muss man die
Skalierungsfragen ebenso gesondert bedenken, wie man seine Make-Prozedur auf die
Vorabnutzung von \acc{lilypond-book} umstellen muss. Und dann bleibt immer noch,
dass man \LaTeX-Konstrukte nicht in \acc{Lilypond}-Umgebungen verwenden kann.

Die \textbf{Vorteile} der \acc{\LaTeX\,+\,\textbf{LilyPond}}-Methode machen sie
jedoch zum Mittel der Wahl:

\begin{itemize}
  \item Zum ersten steht mit \acc{LilyPond} eine im Vergleich zu \acc{Musix\TeX}
  deutlich einfachere Auszeichnungssprache zur Verfügung.
  \item Zum zweiten bietet sich mit \acc{Frescobaldi} ein ausgezeichneter
  semi-graphischen Editor an, der auch die eingebundene Bibliothek.
   \acc{harmonyli.ly} und die daraus genutzten Funktionen korrekt auswertet.
  \item Neben diesem Standardeditor kann man auch auf das gute Eclipse-Plugin
  \acc{elysium} zurückgreifen, wenn man \acc{LilyPond} erstellen will, und zwar
  insbesondere dann, wenn man auch sein \LaTeX-Texte per
  \acc{{\TeX}lipse}-Eclipseplugin erzeugt.
  \item Außerdem darf man hoffen, dass mit \acc{Canorus} noch ein dritter
  Kandidat entsteht, der wenigstens zukünftig gute Dienste leisten kann.
  \item Und schließlich kann man für \acc{LilyPond} auch den graphischen Editor
  \acc{Muse\-Score} als Frontend verwenden, sofern man in Kauf nimmt, vor der
  eigentlichen Arbeit auf \acc{\LaTeX-LilyPond}-Ebene den Konverter
  \acc{musicxml2ly} auf den \acc{MuseScore}-XML-Export anzuwenden, das Ergebnis
  in \acc{Frescobaldi} zu laden und dort die Harmonieanalysesymbole in einem
  zweiten Schritt einzugeben.
\end{itemize}

Damit dürfte auch klar sein, welche Methode wir selbst anwenden werden, nämlich
\acc{\LaTeX\,+\,LilyPond\,+\,harmonyli.ly}. Das wäre nicht möglich gewesen, wenn
es sich hier nicht um Freie Open Source Software gehandelt hätte. Erst das gab
uns die Möglichkeit, die Dinge selbst zu ergänzen, die wir für ein
graphisch und inhaltlich adäquates Ergebnis noch benötigten.

Insgesamt sind wir froh, den Djungel der Möglichkeiten, der sich nach der reinen
Internetrecherche abgezeichnet hatte, gelichtet und die wirklich gangbaren Wege
gefunden zu haben. Jetzt wissen wir, woran wir sind. Wir wünschten allerdings,
wir hätten diesen Text schon eingangs von jemand anderem ausgehändigt bekommen,
anstatt ihn selbst geschrieben haben zu müssen. Das hätte uns sehr viel Zeit
gespart.

Möge also unser Text anderen den Aufwand für eine solche Tool-Evaluation sparen.
Und möge er anderen als Ausgang für Updates, Verfeinerungen und Zusatzarbeiten
dienen. Deshalb sei er auch als CC-BY-SA lizensierter Text veröffentlicht.

Frankfurt 2019-11-26
Karsten Reincke

\chapter{Nachrede: Das 'vergessene Kapitel'?}

Vielleicht mag sich der eine oder andere Leser nun wünschen, den idealen Weg
einmal im Detail vorgeführt zu bekommen. Diesen sei gesagt, dass sie das nicht
mehr nötig haben: Was zur Nutzung von \acc{LilyPond} unter und mit \LaTeX\ zu
sagen war, haben wir im Kapitel über das Backend \acc{LilyPond}
ausgeführt.\footnote{\ra\ \pageref{LilyPondBackend}} Und was zu einer guten
Nutzung der Bibliothek \acc{harmonyli.ly}\footcite[vgl.][\nopage
wp.]{ReinckeBlum2019a} noch zu sagen wäre, bietet bereits dessen
Tutorial.\footcite[vgl.][]{Reincke2019b} Insofern dürfen Sie jetzt direkt
loslegen.

% this is only inserted to eject fault messages in texlipse
%\bibliography{../bib/literature}




% insert the nomenclature here

\input{bib/ncl.abbreviations.tex}
%\input{bib/ncl.journals}
\printnomenclature

% insert the bibliographical data here
\bibliography{bib/literature}

\end{document}
